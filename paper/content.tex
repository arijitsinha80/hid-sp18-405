\title{Stardog}


\author{Min Chen}
\affiliation{%
  \institution{Indiana University}
 \streetaddress{School of Informatics, Computing, and Engineering}
  \city{Bloomington}
  \state{IN}
  \postcode{47408}
  \country{USA}}
\email{mc43@iu.edu}

\begin{abstract}
	Graph databases with RDF data model have been used to represent 
	knowledge with querying and reasoning capabilities. Stardog is a 
	java-based commercial RDF graph database that supports SPARQL query 
	languages, data unification using Virtual Graph and reasoning based on 
	OWL, rules and Integrity Constraints. It provides enriched inference and 
	reasoning beyond the property graph databases with Graph DBMS model 
	and supports integration with cloud technologies such as Amazon Web 
	Service and Pivotal	Cloud Foundry.

\end{abstract}

\keywords{hid-sp18-405, Stardog, Virtual Graph, RDF, Graph Database}

\maketitle

\section{Introduction}

	Stardog is a graph database from US-software company
	Complexible. Stardog has a particular focus on OWL and RDF-based
	systems, and supports SPARQL query language; property graph model and 
	Gremlin graph traversal language; OWL 2 and user-defined rules for 
	inference and data analytic; virtual graphs; and
	programmatic interaction via several languages and network
	interfaces~\cite{hid-sp18-405-www-stardog-docs}. Further, the
	developers of Stardog OWL/RDF DBMS have pioneered a new use of OWL 
	as a schema language for RDF databases. This is achieved by adding
	integrity constraints (IC), also expressed in OWL syntax, to the traditional  
	open-world OWL axioms~\cite{hid-sp18-405-cer2012graphical-stardog}. 
	Other key features of Stardog include Machine Learning and Logical 
	Inference,Semantic Search, Geospatial Search, Integration with Amazon 
	AWS and Pivotal Cloud Foundry etc.\ 

	The technology paper is structured as follows:

		\begin{itemize}
			\item Section~\ref{s:arch} presents the architecture of Stardog 
			Knowledge Graph Platform, which is the integration of Stardog 
			database with the knowledge toolkit.
			
			\item Section~\ref{s:keyf} discusses seven of the key features of 
			Stardog.
			
			\item Section~\ref{s:comp} compares Stardog with two similar graph 
			databases, Neo4j and GraphDB, which are representatives of 
			property graph databases and RDF graph databases respectively.
		
			\item Section~\ref{s:license} and Section~\ref{s:conclusion} 
			summarize the license of Stardog before the conclusion.
		\end{itemize}

\section{Architecture}
\label{s:arch}
	The architecture of Stardog Knowledge Graph Platform, which combines 
	the graph database with knowledge toolkit, is shown in 
	Figure~\ref{sa:archi}.

	\begin{figure}[!ht]
	  \centering\includegraphics[width=\columnwidth]{images/stardog-architecture.png}
	  \caption{Architecture of Stardog Knowledge Graph 
	  Platform~\cite{hid-sp18-405-blog-stardog-kgraph}}\label{sa:archi}
	\end{figure}

	There are three broad components centered around the Stardog graph 
	database within the Knowledge Graph Platform, namely ETL, Virtual and 
	Applications and Analytics. Each component is designed to provide 
	the services in a declarative way. 
	
	\begin{itemize}
		
		\item ETL stands for Extract, Transform, and Load, which are three 
		database functions that are combined to extract data out of one 
		database and insert into another database. Figure~\ref{sa:archi} 
		illustrates that three main types of data: structured, semi-structured 
		and unstructured are extracted and incorporated into the core graph 
		database: Stardog. 
		
		\item Virtual refers to the mapping of relational data into the RDF 
		database as named graphs but without materialization (as in the ETL 
		fashion)~\cite{hid-sp18-405-blog-stardog-virtual}. 
		
		\item Applications and Analytics include generating reports from the 
		database, querying the database and perform analysis using statistical 
		inference and probabilistic reasoning and also built-in machine learning 
		libraries such as Vowpal, Wabbit and 
		XGBoost~\cite{hid-sp18-405-blog-stardog-ml}~\cite{hid-sp18-405-blog-stardog-xgboost}.
	
\end{itemize}

\section{Key Features}
\label{s:keyf}
	In this section, the author discusses several key features of Stardog 
	including Virtual Graph, Integrity Constraints, OWL and Rule 
	Reasoning, Stardog Studio, Machine Learning, High Availability Cluster, 
	Integration with AWS and PCF etc.\

	\subsection{Virtual Graph}
		Virtual Graph is a feature that facilitates the mapping of relational data 
		into the RDF databases. ``Stardog supports the standard W3C R2RML 
		mapping language~\cite{hid-sp18-405-www-stardog-r2rml} for defining 
		how data in a relational system maps to RDF graphs'' and ``the mapped 
		triples representing the source relational data are considered to be in a 
		named graph that is not present (i.e., not materialized) in the local RDF 
		graph''~\cite{hid-sp18-405-blog-stardog-virtual}, hence, the named 
		graph is considered virtual.
		
		When dealing with unified data sources, users could either apply ETL 
		(Extract, Transform, and Load) after materialization of the virtual graph 
		or 	directly query the virtual graph using federated queries (virtual 
		queries). Federated query performs a translation of a SPARQL query into 
		a SQL query and the execution is through a relational database 
		engine~\cite{hid-sp18-405-blog-stardog-virtual2}~\cite{hid-sp18-405-diego2017ontop-stardog}.
		Key trade-offs between these two operational models are summarized 
		as follows:
		
		 \begin{itemize}
		 	
			\item Evaluation of queries over materialized data via ETL does not 
			involve any communication with the source system. This in general 
			leads to better query performance and independence of queries from 
			the availability of the source 
			system~\cite{hid-sp18-405-blog-stardog-virtual}.
			
			\item Materialization, on the other hand, takes multiple steps and 
			resources for creating and storing tuples in RDF model, which is 
			time-consuming. Further, when the data points are modified 
			frequently before queried, materialization will lead to a worse 
			performance compared to virtual queries, which is essentially 
			real-time reasoning.
			
		\end{itemize}
	
		Stardog offers both ways of unifying data, federated queries and 
		materialization. The system allows users to switch between the two and 
		the ``choices can be made on a source by source 
		basis''~\cite{hid-sp18-405-blog-stardog-virtual}.
	
	\subsection{Integrity Constraints}
		In Stardog, Integrity Constraints (IC) are used to validate RDF data 
		based on constraints or rules imposed by the database users. Stardog 
		supports multiple languages for specifying the rules including SPARQL 
		and OWL, which allows querying and mapping these rules in SPARQL as 
		well~\cite{hid-sp18-405-www-stardog-docs}. Implementation of such 
		constraints allows the users to apply domain-specific knowledge to the 
		data and align the knowledge with RDF.\@ Integrity Constraints can then 
		be utilized in the reasoning procedure to ensure logical consistency and 
		explain errors, which is the advantage of RDF database over plain 
		property graph database in general. 

	\subsection{OWL and Rule Reasoning}
		Stardog’s OWL reasoning is based on the OWL 2 Direct Semantics 
		Entailment Regime and Stardog performs reasoning at query time 
		without inference materialization. In addition, Stardog provides 
		explanation of an inference by ``minimum set of statements explicitly 
		stored in the database that, together with the schema and any valid 
		inferences, logically justify the 
		inference''~\cite{hid-sp18-405-www-stardog-docs}. Under the 
		circumstances where OWL’s axiom-based approach is not adequate for 
		the reasoning, Stardog allows User-defined rules as a complements and 
		enhance the power of the reasoning by combining both OWL and rules 
		into the system. 

	\subsection{Stardog Studio}
		Stardog Studio-the Knowledge Graph IDE, which is announced early 
		2018, is a front end developing tool for Stardog. It includes a SPARQL 
		query notebook, which provides ``syntax highlighting, prefix 
		auto-completion, and exporting 
		results''~\cite{hid-sp18-405-blog-stardog-studio}. In addition, users 
		could also ``execute SPARQL queries against Stardog database and 
		view results inside Stardog studio and export query results to the file 
		system''~\cite{hid-sp18-405-www-stardog-studio}.
		
		Stardog Studio also provides the functionality of database management 
		and security view. These allow the users to view and administer the 
		Stardog databases as well as user, role and permissions for the Stardog 
		system~\cite{hid-sp18-405-blog-stardog-studio}.
		
		Additional features like visualization and cluster management tools are 
		under development and expected in future 
		releases~\cite{hid-sp18-405-www-stardog-studio}.

	\subsection{Machine Learning}
		With the built-in machine learning libraries such as Vowpal, Wabbit and 
		XGBoost, Stardog could perform traditional machine learning with 
		statistical	inference and probabilistic 
		models~\cite{hid-sp18-405-blog-stardog-xgboost}. 
		
		Further, Machine Learning has been used in two unique ways 
		supporting the knowledge graph. First, learning methods and algorithms 
		are applied when creating  the Knowledge Graph which unifies different 
		data sources. Second,  machine learning is also utilized to obtain 
		actionable insight from the data unified. For example, predictive 
		analytics is used to predict nodes and edges in a Knowledge Graph, and 
		extract patterns from the data in order to make forecast based on those 
		patterns~\cite{hid-sp18-405-blog-stardog-ml}.


	\subsection{High Availability Cluster}
		Stardog utilizes High Availability Clusters for uninterrupted operations, 
		redundancy and high query 
		volume~\cite{hid-sp18-405-www-stardog-docs}. 
		The clusters aims at mitigating the risk of failure on a single machine by 
		automatically creating multiple copies of the service with Apache 
		ZooKeeper as the distributed coordination 
		tool~\cite{hid-sp18-405-www-stardog-predictiveanalyticstoday}. The 
		cluster size affects performance in two ways: larger cluster sizes 
		perform better for reads and perform worse for writes compared to 
		small cluster sizes~\cite{hid-sp18-405-www-stardog-docs}. 

	\subsection{Integration with AWS and PCF}
		The Stardog High Availability Cluster supports installation by Stardog 
		Graviton, which complies to a single binary executable. This facilitates 
		the integration with Amazon Web Services and users could easily 
		deploy, configure, and launch a Stardog cluster on Amazon 
		AWS~\cite{hid-sp18-405-blog-stardog-aws}. 
		
		Besides, ``the integration with Pivotal enables applications running in 
		Pivotal Cloud Foundry to natively connect to Stardog instances without 
		having to manually wire apes to 
		services''~\cite{hid-sp18-405-blog-stardog-pcf}. This has been made 
		available by the announcement of Stardog Service Broker for Pivotal 
		Cloud Foundry.

\section{Comparison with Related Technologies}
\label{s:comp}
	Stardog is a graph database with RDF as a primary data model. Besides 
	Stardog, there are other leading RDF graph databases including 4Store, 
	GraphDB and Sesame. On the contrary, there is another type of graph 
	databases, often referred to as property graph, that applies general Graph 
	DBMS model without RDF.\@ Neo4j is one of the leading technology in this 
	category. In this section, Stardog will be compared to GraphDB and Neo4j, 
	illustrating strengths and weaknesses of Stardog both within the category 
	of RDF database and against the other category namely property graph 
	database. A comparison of some of the system properties of the three 
	graph databases are summarized in 
	Table\ref{t:comparison}

	\begin{table}[htb]
		\centering
		\caption{System Properties Comparison GraphDB vs.\ Neo4j vs.\ 
		Stardog~\cite{hid-sp18-405-www-stardog-dbengines-neo4j}~\cite{hid-sp18-405-www-stardog-dbengines-graphdb}}\label{t:comparison}
		\begin{tabular}{llll}
		Name & GraphDB & Neo4j & Stardog \\
		\toprule
		Database model&	Graph DBMS and RDF & Graph DBMS & Graph DBMS 
		and RDF \\
		\midrule
		DB-Engines Ranking (Graph DBMS) &\#12 &\#1 &\#10\\
		\midrule
		DB-Engines Ranking (RDF) &\#7 &N/A &\#6\\
		\midrule
		Developer & Ontotext &	Neo4j, Inc.	& Complexible Inc.\\
		\midrule
		Initial release	&2000 &	2007&	2010\\
		\midrule
		License &	commercial &	Open Source &	commercial \\
		\midrule
		Implementation language &	Java &	Java, Scala &	Java\\
		\midrule
		Any SQL supported &	SPARQL &	no	& SPARQL \\
		\midrule
		In-memory capabilities&no &no &			yes\\
		\midrule
		XML support&	no	& no& 	partially\\
		\bottomrule
		\end{tabular}
	\end{table}

	\subsection{Stardog vs.\ Neo4j}
		Neo4j, as a leading property graph database (ranking \#1 by DB-Engines 
		according to Table\ref{t:comparison}), has strength in the following 
		aspects. First, it is highly flexible that most objects and relations could 
		be represented as nodes and edges respectively in the graph. Second, it 
		does not require schema or ontology and thus light-weighted compared 
		to RDF databases. Finally, it has a relative simple graph structure for 
		traversals and 
		analysis~\cite{hid-sp18-405-robinson2013graphdatabase-stardog}. 
		
		However, there are several important features of Stardog that property 
		graph like Neo4j could not achieve. First, Neo4j only supports 
		materialization of data. Virtualization of data (virtual graph) cannot be 
		performed. Second, query language used by property languages such 
		as Cypher and Gremlin lack the expressibility and ability to yield 
		structured views of data compared to query languages like 
		SPARQL~\cite{hid-sp18-405-angles2008expre-stardog}. 
		Stardog uses SPARQL as a main query language and also supports all of 
		Apache TinkerPop3 including 
		Gremlin~\cite{hid-sp18-405-www-stardog-docs}. Finally, without RDF 
		and OWL, property graph cannot impose integrity constraints, 
		explanations, user-defined rules or reasoning, which are all achievable in 
		Stardog~\cite{hid-sp18-405-www-stardog-dbengines-neo4j}~\cite{hid-sp18-405-www-stardog-docs}.
	 
	\subsection{Stardog vs.\ GraphDB}
		Both Stardog and GraphDB support RDF models and share many 
		important features including reasoning, user-defined rules, SPARQL 
		query and machine learning modules. However, Stardog has the 
		capability of Virtual Graph which avoids materialization when unifying 
		data sources, which is a key strength compared to GraphDB.\@ 
		GraphDB on the other hand, has a major advantage and focus on 
		Natural Language Processing (NLP) and text mining by providing 
		Ontotext Platform as an integrated text analysis 
		system~\cite{hid-sp18-405-www-stardog-ontotext}, while Stardog only 
		supports text analytics indirectly by providing connectors to other NLP 
		libraries OpenNLP~\cite{hid-sp18-405-www-stardog-docs}. 
		
		Besides capabilities, researchers have been testing and comparing the 
		performance of RDF graphs including GraphDB and Stardog. Based on 
		experiments on real data, Ledvinka, Martin and K{\v{r}}emen concluded 
		that ``GraphDB, (and storages performing materialization in general) 
		has a major disadvantage in that the user has to specify inference level 
		before actually inserting data into the storages. Real time reasoning (like 
		Stardog), on the other hand, lets the user choose reasoning level at the 
		query time. However, GraphDB appears to be more suitable for the 
		object-oriented application access scenario, in which frequent data 
		updates are 
		expected''~\cite{hid-sp18-405-ledvinka2015object-stardog}. In a more 
		recent study, Luyen and his colleagues compared six RDF data models: 
		4Store, Virtuoso, Stardog, GraphDB, Sesame and Jena Fuseki (TDB) 
		using large RDF graphs. They found that Stardog gives the best results 
		compared to the criteria: Data Loading, Data Search and Data Inference 
		therefore they stated that in general outperforms the other five 
		candidates for their 
		Benchmark~\cite{hid-sp18-405-luyen2016development-stardog}.


\section{License}
\label{s:license}
	As a commercial software, Stardog is priced for community, developer and 
	enterprise tiers. The community version is free with 10 databases, 25 
	million triples per database and the developer version offers a free 30-day 
	trial with unlimited data or 
	machines~\cite{hid-sp18-405-www-stardog-predictiveanalyticstoday}. The 
	enterprise version comes with a server management module and customer 
	support by both phone and email~\cite{hid-sp18-405-www-stardog-docs}.


\section{Conclusion}
\label{s:conclusion}
	Stardog, as a commercial RDF-based graph database, supports data 
	unification using both materialization and virtualization methods, and 
	allows semantic reasoning and logical inferences by utilizing integrity 
	constraints, OWL, and user-defined rules. The advantages of virtual 
	graph, SPARQL query, and reasoning capability has made it an alternative 
	to property graph databases, such as Neo4j. 


\begin{acks}

  The author would like to thank Dr.~Gregor~von~Laszewski for his
  support and suggestions to write this paper.

\end{acks}

\bibliographystyle{ACM-Reference-Format}
\bibliography{report} 

