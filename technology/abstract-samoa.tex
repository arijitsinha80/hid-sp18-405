\section{Apache Samoa}
\index{Apache!Samoa}
\index{Samoa}

Apache Samoa, which stands for Scalable Advanced Massive Online
Analysis, is a distributed streaming machine learning framework that
contains a programming abstraction for distributed streaming machine
learning algorithms~\cite{hid-sp18-405-www-samoa}. ``It features a
pluggable architecture that allows it to run on several distributed
stream processing engines such as Storm, S4, and
Samza''~\cite{hid-sp18-405-www-samoa}. Real time analytics can be
utilized by tools like Samoa and allow organizations to react in a
timely manner when problems appear or to detect new trends helping to
improve their performance by obtain useful knowledge from what is
happening now~\cite{hid-sp18-405-bif2015mining-samoa}. Apache Samoa
users can develop distributed streaming ML algorithms once and execute
them on multiple DSPEs (distributed stream processing
engine)~\cite{hid-sp18-405-mor2015samoa-samoa}. In addition, users
could also add new platforms by using the API provided, therefore, the
Samoa project is divided into two different parts, namely: Samoa-API and 
Samoa-Platform. By using Samoa-API, developers could develop for Samoa 
without worrying about which DSPE is going to be 
used~\cite{hid-sp18-405-blog-samoa}. Samoa, written in Java, is open 
source under the Apache Software License version 2.0.

